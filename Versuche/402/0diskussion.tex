%
%
\subsection{Statistische Fehler}
Die Messung lies einige statistische Fehler zu:

1. Die Apparatur verwackelt bei Drehung des Fernrohres und des Prisma.

2.Die Spektrallinien können nicht als idealisiert betrachtet werden, da man einen Spalt
endlicher Breite hat.

3. Der Raum war nicht richtig abgedunkelt bzw. die im Raum befindlichen Lichtquellen,
wie Leselampen, haben den Versuch beeinflusst.

4. Auf der Apparatur befindet sich teilweise Staub. Sowohl auf dem Prisma, als auch auf
den optischen Bauteilen des Fern- und Kollimatorrohres.

Die Messung des Winkels $\varphi$ zwischen den Oberflächen war mit $59,96^\circ \pm 0,04^\circ$ im Bereiches der erwarteten $60^\circ$ des Prismas, Abweichungen lassen sich auf statistische Fehler beim Messvorgang zurückführen.
Zusätzlich gab es nur wenige Punkte die ausgewertet werden konnten, da die Lichtverhältnisse mehr nicht zuließen. Auch dies führte zu weiteren Ungenauigkeiten.
\subsection{Grundsätzliche Ungenauigkeiten}
Neben diesen apparaturbedingten Fehlern ist ein grundsätzlicher Fehler unterlaufen, als ein gespiegeltes 
Spektrum ausgemessen wurde. Dies ließ sich jedoch aufgrund einfacher geometrischer Zusammenhänge wieder auf das eigentlich
zu messende Spektrum schließen. \\
Die Abbesche Zahl,
\begin{align}
v&=50\pm10,
\end{align}
sowie das theoretische Auflösungsvermögen für $\lambda_C$ und $\lambda_F$,
\begin{align}
A_C&=1530\pm80,\text{ }A_F=3400\pm500,
\end{align}
sind verhältnismäßig ungenau. Den ebenso wie die Berechnung der Absroptionsstelle,
\begin{align}
\lambda_i&=(114\pm 6)\text{ nm},
\end{align}
 beruhen die Werte auf den Anfangs errechneten Werten der auch nur genäherten Dispersionskurve. 
 Wenn diese Anfangswerte nun aber aus ungenau gemessenen, teilweise erst noch gespiegelten Messwerten 
 errechnet worden sind, so sind darauf folgende Werte mit einem mindestens gleich großen Fehler behaftet. 