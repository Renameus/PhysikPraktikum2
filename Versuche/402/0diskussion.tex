%
%
Die Messung lies einige statistische Fehler zu:

1. Die Apparatur verwackelt bei Drehung des Fernrohres und des Prisma.

2.Die Spektrallinien können nicht als idealisiert betrachtet werden, da man einen Spalt
endlicher Breite hat.

3. Der Raum war nicht richtig abgedunkelt bzw. die im Raum befindlichen Lichtquellen,
wie Leselampen, haben den Versuch beeinflusst.

4. Auf der Apparatur befindet sich teilweise Staub. Sowohl auf dem Prisma, als auch auf
den optischen Bauteilen des Fern- und Kollimatorrohres.

Die Messung des Winkels $\varphi$ zwischen den Oberflächen war mit $60,18^\circ \pm 0,58^\circ$ im Bereiches der erwarteten $60^\circ$ des Prismas, abweichen lassen sich auf statistische Fehler beim Messvorgang zurückweisen.
Jedoch wurden bei $\eta$ und n zu kleine abweichungen gemessen, auch dies vermutlich aufgrund von statistischen Fehlern beim messen. Diese sollten laut der Literatur größer sein, dies führte zu Ungenauigkeiten. Dazu gab es nur wenige Punkte die ausgewertet werden konnten, da die Lichtverh"altnisse mehr nicht zulie"sen. Auch dies f"uhrte zu weiteren Ungenauigkeiten. Bei der Dispersionskurve wurde die steigung von $A_2$ bei $n^2(\lambda)$ negativ, dies darf laut Versuchsanleitung nicht geschehen.

Aus diesem Grund weist die Dispersionskurve sowie die Abbesche Zahl grobe Fehler auf und sollte nicht weiterverwendet werden.