% http://fachschaft.physik.uni-dortmund.de/images/GlobaltutAP/protokoll.txt
% ========================================
%	Header einbinden
% ========================================

\input{apheaderneu.tex}
\renewcommand*\rmdefault{iwona}\normalfont\upshape

% ========================================
%	Angaben für das Titelblatt
% ========================================

\title{Versuch 402 - Dispersionsmessungen am Glasprisma\\				% Titel des Versuchs 
\large TU Dortmund, Fakultät Physik\\ 
\normalsize Anfänger-Praktikum}

\author{Oliver Zietek\\			% Name Praktikumspartner A
{\small \href{oliver.zietek@tu-dortmund.de}{oliver.zietek@tu-dortmund.de}}	% Erzeugt interaktiven einen Link
\and						% um einen weiteren Author hinzuzfügen
Fabian Lehmann\\					% Name Praktikumspartner B
{\small \href{fabian.lehmann@tu-dortmund.de}{fabian.lehmann@tu-dortmund.de}}		% Erzeugt interaktiven einen Link
}


\date{20. Dezember 2012}				% Das Datum der Versuchsdurchführung

% ========================================
%	Das Dokument beginnt
% ========================================

\begin{document}

% ========================================
%	Titelblatt erzeugen
% ========================================

\maketitle					% Jetzt wird die Titelseite erzeugt
\thispagestyle{empty} 				% Weder Kopfzeile noch Fußzeile

% ========================================
%	Der Vorspann
% ========================================

%\newpage					% Wenn Verzeichnisse auf einer neuen Seite beginnen sollen
%\pagestyle{empty}				% Weder Kopf- noch Fußzeile für Verzeichnisse

\tableofcontents

%\newpage					% eine neue Seite
%\thispagestyle{empty}				% Weder Kopf- noch Fußzeile für Verzeichnisse
%\listoffigures					% Abbildungsverzeichnis

%\newpage					% eine neue Seite
%\thispagestyle{empty}				% Weder Kopf- noch Fußzeile für Verzeichnisse
%\listoftables					% Tabellenverzeichnis
\newpage					% eine neue Seite


% ========================================
%	Kapitel
% ========================================

%\section{Einleitung}				% Bei Bedarf
%	\input{0einleitung.tex}
\section{Theorie}
	%picdispkurve,picapparatur,picsymstrahl,picprismawinkel,picbrechwinkel
%eqbrech, eqdispcurve, eqdisp1, eqdisp2
\subsection{Brechung und Dispersion}
	\begin{figure}[h]
		\begin{center}
		\includegraphics[scale=0.7]{picdispkurve.jpg}
		\caption{Gestalt möglicher Dispersionskurven[1]}
		\label{picdispkurve}
		\end{center}	
	\end{figure}
Durch Wechselwirkungen mit Elektronen ist die Geschwindigkeit von Licht in Materie
geringer als die Vakuumslichtgeschwindigkeit. Dadurch kommt es zu Brechungen an 
Materiegrenzflächen nach Gleichung (\ref{eqbrech}) mit dem Brechungsindex $n$.
Da die Ausbreitungsgeschwindigkeit von der Wellenlänge $\lambda$ des Lichtes abhängt, 
wird Licht unterschiedlicher Frequenz verschieden stark gebrochen. Dieses
Phänomen wird Dispersion genannt. Eine Abhängigkeit nach (\ref{eqdispkurve})
heißt Dispersionskurve\cite{anleitung} (vgl. Abb. \ref{picdispkurve}).
\begin{align}
n&=\frac{v_1}{v_2} \label{eqbrech} \\
v_1&, v_2 : \text{materialabhängige Geschwindigkeit von Licht} \nonumber \\
n&=f(\lambda) \label{eqdispkurve}
\end{align}
Aus dem Huygensschen Prinzip lässt sich das Snelliussche Brechungsgesetz (Gl. (\ref{eqsnellius})) folgern,
bei einem Eintritt unter einem Winkel $\alpha$ folgt mit dem Brechungsindex $n$ ein
Austrittswinkel $\beta$.
\begin{align}
n&=\frac{sin \alpha}{sin \beta}\label{eqsnellius}
\end{align}
\subsection{Dispersion an einem Prisma}
Unter Annahme eines Strahlendurchgangs von sichtbarem Licht von Luft in
ein durchsichtiges, farbloses Material lassen sich Dispersionsgleichungen \cite{anleitung} ableiten:
Gilt für die Absorbtionsstelle $\lambda_1$ mit der Wellenlänge $\lambda$ im Vakuum, dass $\lambda>>\lambda_1$,
so gilt Gleichung (\ref{eqdisp1}), welche in Abbildung \ref{picdispkurve} als a) dargestellt ist.
Gilt hingegen $\lambda<<\lambda_1$, so gilt Gleichung (\ref{eqdisp2}), welche in Abbildung \ref{picdispkurve} 
als b) dargestellt ist. Beide Kurven beschreiben normale Dispersion, also Abnahme von $n$ bei Zunahme von $\lambda$.
\begin{align}
n^2(\lambda)&=A_0+\frac{A_2}{\lambda^2}+\frac{A_4}{\lambda^4}+\ldots \text{ mit } A_i>0 \label{eqdisp1} \\
n^2(\lambda)&=1-A_2' \lambda^2-A_4'\lambda4-\ldots \text{ mit } A_i'>0, i\geq 2 \label{eqdisp2}
\end{align}
\subsection{Prismenspektralapparat}
	\begin{figure}[h]
		\begin{center}
		\includegraphics[scale=0.3]{picapparatur.jpg}
		\caption{Schematische Darstellung des Prismenspektralapparates[1]}
		\label{picapparatur}
		\end{center}	
	\end{figure} 	\begin{figure}[h]
		\begin{center}
		\includegraphics[scale=0.4]{picsymstrahl.jpg}
		\caption{Symmetrischer Strahlengang durch ein Prisma[1]}
		\label{picsymstrahl}
		\end{center}	
	\end{figure}
Mit Hilfe eines Prismenspektralapparates (siehe Abb. \ref{picapparatur}) lässt sich der Brechungsindex 
abhängig von der Wellenlänge durch ABlesen von Winkeln bestimmen. Das Gerät macht sich das Snelliussche Brechungsgesetz zu nutze,
bei dem symmetrischen Durchgang des Lichtstrahls (Abb. \ref{picsymstrahl}) lässt sich folgende Gleichung 
herleiten.
\begin{align}
n&=\frac{sin\frac{\eta + \phi}{2}}{\frac{\phi}{2}}\\
\text{ mit }\alpha&=\frac{\eta + \phi}{2}
\end{align}
\subsection{Bestimmung des Brechungswinkels zwischen brechenden Oberflächen des Prismas}
	\begin{figure}[h]
		\begin{center}
		\includegraphics[scale=0.3]{picprismawinkel.jpg}
		\caption{Bestimmung des Winkels zwischen den brechenden Oberflächen[1]}
		\label{picprismawinkel}
		\end{center}	
	\end{figure}
Wird das Prisma wie in Abbildung \ref{picprismawinkel} ausgerichtet, kann aus einfachen
Winkelbeziehungen durch Messen der Reflektionswinkel beider Lichtstrahlen der brechende 
Winkel $\phi$ des Prismas bestimmt werden.
\begin{align}
\phi&=\frac{1}{2}(\phi_r - \phi_l)
\end{align}
\subsection{Bestimmung der Brechungswinkel der Spektrallinien}
	\begin{figure}[h]
		\begin{center}
		\includegraphics[scale=0.3]{picbrechwinkel.jpg}
		\caption{Brechwinkelbestimmung mit spiegelbildlicher Prismenstellung[1]}
		\label{picbrechwinkel}
		\end{center}	
	\end{figure}
Aus zwei zueinander spiegelsymmetrischen Stellungen des Prismas können die Brechungswinkel
bestimmt werden. Bei dem Zusammenfallen der gebrochenen Lichtstrahlen mit dem reflektierten
Lichtstrahl lässt sich mit Abbildung \ref{picbrechwinkel} folgende Beziehung ableiten.
\begin{align}
\eta&=180-(\Omega_r-\Omega_l)
\end{align}
	\FloatBarrier
\section{Durchführung}
	%
%
\subsection{Bestimmung des Brechungswinkels zwischen brechenden Oberflächen des Prismas}
Das Prisma wird mit der brechnenden Kante ungefähr auf das Kollimatorrohr ausgerichtet (vgl. Abb. \ref{picprismawinkel}).
Dann werden die Reflektionswinkel mit dem Fernrohr gemessen. Dieser Vorgang wird mehrfach für die verschiedenen Kanten wiederholt.
\subsection{Bestimmung der Brechungswinkel der Spektrallinien}
Das Prisma wird so lange gedreht, bis die jeweilige Farbe des gebrochenen Lichtspektrums mit dem reflektierten Lichtstrahl
zusammenfällt. Dann wird das Prisma in eine spiegelsymmetrische Stellung gebracht und der Vorgang wird wiederholt (vgl. Abb. \ref{picbrechwinkel}).
	\FloatBarrier
\section{Auswertung}
	\subsection{Bestimmung der Brechungsindices}

Zur Messung des Brechnungsindex benutzen wir einen Prismenspektralapparat.
Dabei ist ein Glasprisma auf dem Drehteil eines Goniometers montiert. Durch
einen Spalt und eine Sammellinse fällt Licht aus einer Heliumlampe auf
das Prisma. Dadurch, dass der Spalt in der Brennebene der Linse steht, erhalten
wir paralleles Licht. Nach der Brechung fällt das Licht in die Brennebene einer Lupe, durch welche wir
die Spektrallinien sehen können.
Da wir nur den symmetrischen Fall betrachten, bei dem a = a' und b = b'
ist, folgt
\begin{align}
n=\frac{sin(\frac{\varphi + \eta}{2})}{sin(\frac{\varphi}{2})}\nonumber
\end{align}

Um sicherzustellen, dass ein symmertrischer Strahlenverlauf vorliegt, bringt
man den gebrochenen Strahl mit dem an der Basis des Prismas reflektrierten
Strahl zur Deckung. Dann misst man die Auslenkung des gebrochenen Strahles
links und rechts und berechnet den Brechnungswinkel wie folgt:

\begin{align}
-\eta=180^\circ +\Omega_l - \Omega_r \nonumber
\end{align}

Weiterhin sollte der Winkel an der brechenden Kante gemessen. Dafür wurde
diese Kante auf das Kollimatorrohr gerichtet und der Winkel des reflektierten
Strahles gemessen. Dieser Winkel wurde auf beiden Seiten fünf mal abgelesen,
wobei das Prisma mit der brechenden Kante über den Strahl hinweg auf eine
spiegelverkehrte Position gedreht wurde. Damit ergibt sich mit der Formel:

\begin{align}
\varphi=\frac{1}{2} (\varphi_r - \varphi_l) \nonumber
\end{align}

\begin{table}[h]
\begin{center}
\begin{tabular}[c]{|c|c|c|} \hline
$\varphi_l$ in $^\circ$ & $\varphi_r$ in $^\circ$ & $\varphi$  in $^\circ$\\ \hline
263,9 & 383,6 & 59,9 \\ 
229,2 & 349,2 & 60,0 \\ 
229,5 & 349,3 & 59,9 \\ 
241,4 & 361,5 & 60,1 \\ 
240,0 & 360,0 & 60,0 \\ \hline
\end{tabular}
\caption{Winkel $\Phi$ an der brechenden Kante des Prismas}
\end{center}
\end{table}
% \newpage

Somit lautet der Mittelwert:
$\varphi = 59,96^\circ \pm 0,04^\circ$ Der Fehler wird mit der Formel der Standardabweichung des Mittelwertes
berechnet.

Für die Indicies n und $\eta$ folgt damit:
\begin{table}[h]
\begin{center}
\begin{tabular}[c]{ccccccc}
Farbe & $\lambda$ in nm & $\Omega_l$  in $^\circ$ & $\Omega_r$  in $^\circ$ & $\eta$  in $^\circ$ & n & $\Delta$ n \\ \hline
Rot & 706,3 & 147,5 & 386,2 & 58,7 & 1,721&  0,028\\
Gelb & 578,4 & 147,3 & 386,7 & 59,4 & 1,727&  0,028\\
Gr"un & 501,5 & 146,8 & 387,2 & 60,4 & 1,736& 0,028\\
Blau & 447,1 & 146,4 & 387,5 &  61,1 & 1,742& 0,029\\
\end{tabular}
\caption{Brechungsindicies n und $\eta$}
\end{center}
\end{table}

\subsection{Bestimmung der Dispersionskurve}

Um die Dispersionskurve zu bestimmen, werden die Quadrate der Brechungsindize
gegen Quadrate der Wellenlängen und einmal gegen die Kehrwerte der Wellenlängenquadrate aufgetragen.
Somit kann man eine linearen Regression y = mx + b bestimmen. Durch die Abweichung der
Kurve von den Messwerten kann bestimmt werden, welche Dispersionskurve \cite{anleitung} die richtige ist,
\begin{align}
n^2(\lambda)&=A_0+\frac{A_2}{\lambda^2}+\ldots \label{eqdisp11o} \\
\text{oder } n^2(\lambda)&=A_0'-A_2'\lambda^2-\ldots \label{eqdisp11ao}\text{ }.
\end{align}
\begin{table}[h]
	\begin{center}
		\begin{tabular}{ccc}
			$\text{n}^2$&$\lambda^2\cdot 10^{23}/\text{ m}^2$&$1/\lambda^2\cdot 10^{-24}/(1/\text{ m}^2)$ \\ \hline
			2,96&4,99&2,00\\
			2,98&3,35&2,99\\
			2,98&3,35&2,99\\
			3,01&2,52&3,98\\
			3,04&2,00&5,00
		\end{tabular}
		\caption{Basiswerte für die Methode der kleinsten Quadrate}
		\label{tabwerteo}
	\end{center}
\end{table}
Aus Tabelle \ref{tabwerteo} lassen sich die Gleichungen wie folgt bestimmen,
\begin{align}
n^2(\lambda)&=(2,91\pm0,01)+\frac{(2,5\pm0,1)\cdot 10^4}{\lambda^2} \\
\text{oder } n^2(\lambda)&=(3,08\pm0,02)-(2,4\pm0,5)\cdot 10^{-7}\lambda^2.
\end{align}
Daraus lässt sich über die Bestimmung der Summe der Abweichungsquadrate\cite{anleitung} die richtige
Dispersionsgleichung bestimmen.
\begin{align}
s_n^2&=\frac{1}{z-2}\sum_{i=1}^z\left( n^2(\lambda_i - A_0 - \frac{A_2}{\lambda_i^2})\right)^2= 9,38\cdot 10^{-6}\\ 
s_{n'}^2&=\frac{1}{z-2}\sum_{i=1}^z\left( n^2(\lambda_i - A_0' + A_2'\lambda_i^2)\right)^2= 5,20\cdot 10^{-2}\\ 
Z=4 \text{ Anzahl der Messwertepaare}
\end{align}
Da Gleichung (\ref{eqdisp11o}) das kleinere $s^2$ besitzt, handelt es sich dabei um die hier gültige Gleichung, 
welche in Abbildung \ref{Dispersionskurve} dargestellt ist.
\begin{figure}[h]
	\centering
		\includegraphics[width=1.00\textwidth]{Plot1B.png}
		\caption{Dispersionskurve}
	\label{Dispersionskurve}
\end{figure}

\subsection{Bestimmung der Abbesche Zahl}
Zur Berechung der Abbeschen Zahl benötigt man die Formel
\begin{align}
v=\frac{n_D-1}{n_F-n_C}\nonumber
\end{align}
Für die man die Brechungsindizes der Fraunhofer'schen Linien benötigt. Diese erhält man
indem man $\lambda_C$= 656nm,$\lambda_D$ = 589nm und $\lambda_F$ = 486nm in die Gleichung (\ref{eqdisp11o}) einsetzt.

\begin{align}
n_C= (1,72\pm0,002)\nonumber
\end{align}
\begin{align}
n_D= (1,73\pm0,002)\nonumber
\end{align}
\begin{align}
n_F= (1,74\pm0,002)\nonumber
\end{align}
\begin{align}
v= (53\pm11) \nonumber
\end{align}

\subsection{Auflösungsvermögen}
Das theoretische Auflösungsvermögen $A$ bei voll ausgeleuchtetem Prisma mit einer Basisbreite von
$b=3$ cm  lässt sich mit Gleichung (\ref{eqaufo}) mit Gleichung (\ref{eqdisp11o}) für $\lambda_F$ und $\lambda_C$ berechnen \cite{anleitung}.
\begin{align}
A&=\frac{\lambda}{\Delta \lambda}=b\frac{\text{d}n}{\text{d}\lambda} \label{eqaufo}\\
n(\lambda)&=\left(A_0+\frac{A_2}{\lambda^2} \right)^{1/2}\\
\frac{\text{d}n}{\text{d}\lambda}&=A_2 \lambda^{-3} \left( A_0+\frac{A_2}{\lambda^2}\right)^{-1/2} \\
\Rightarrow A_C&=1530\pm80,\text{ }A_F=3400\pm500
\end{align}

\subsection{Absorptionsstelle}
Die dem sichtbaren Licht nächst gelegene Absorptionsstelle $\lambda_i$ lässt sich berechnen, wenn in Gleichung
(\ref{eqdisp11o}) $n=1$ gesetzt wird.
\begin{align}
\lambda_i&=\left( \frac{A_2}{A_0-1} \right)^{1/2}=(114\pm 6)\text{ nm}
\end{align}





	\FloatBarrier
\section{Diskussion}
	%
%
Die Messung lies einige statistische Fehler zu:

1. Die Apparatur verwackelt bei Drehung des Fernrohres und des Prisma.

2.Die Spektrallinien können nicht als idealisiert betrachtet werden, da man einen Spalt
endlicher Breite hat.

3. Der Raum war nicht richtig abgedunkelt bzw. die im Raum befindlichen Lichtquellen,
wie Leselampen, haben den Versuch beeinflusst.

4. Auf der Apparatur befindet sich teilweise Staub. Sowohl auf dem Prisma, als auch auf
den optischen Bauteilen des Fern- und Kollimatorrohres.

Die Messung des Winkels $\varphi$ zwischen den Oberflächen war mit $60,18^\circ \pm 0,58^\circ$ im Bereiches der erwarteten $60^\circ$ des Prismas, abweichen lassen sich auf statistische Fehler beim Messvorgang zurückweisen.
Jedoch wurden bei $\eta$ und n zu kleine abweichungen gemessen, auch dies vermutlich aufgrund von statistischen Fehlern beim messen. Diese sollten laut der Literatur größer sein, dies führte zu Ungenauigkeiten. Dazu gab es nur wenige Punkte die ausgewertet werden konnten, da die Lichtverh"altnisse mehr nicht zulie"sen. Auch dies f"uhrte zu weiteren Ungenauigkeiten. Bei der Dispersionskurve wurde die steigung von $A_2$ bei $n^2(\lambda)$ negativ, dies darf laut Versuchsanleitung nicht geschehen.

Aus diesem Grund weist die Dispersionskurve sowie die Abbesche Zahl grobe Fehler auf und sollte nicht weiterverwendet werden.
	\FloatBarrier
% ========================================
%	Literaturverzeichnis
% ========================================
\nocite{link1} \nocite{link2}
\bibliographystyle{plainnat}			% Bibliographie-Style auswählen
\bibliography{lit}			% Literaturverzeichnis

% ========================================
%	Das Dokument endent
% ========================================
%\includegraphics[scale=0.75]{img011.jpg}
\end{document}