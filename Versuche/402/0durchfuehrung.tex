%
%
\subsection{Bestimmung des Brechungswinkels zwischen brechenden Oberflächen des Prismas}
Das Prisma wird mit der brechnenden Kante ungefähr auf das Kollimatorrohr ausgerichtet (vgl. Abb. \ref{picprismawinkel}).
Dann werden die Reflektionswinkel mit dem Fernrohr gemessen. Dieser Vorgang wird mehrfach für die verschiedenen Kanten wiederholt.
\subsection{Bestimmung der Brechungswinkel der Spektrallinien}
Das Prisma wird so lange gedreht, bis die jeweilige Farbe des gebrochenen Lichtspektrums mit dem reflektierten Lichtstrahl
zusammenfällt. Dann wird das Prisma in eine spiegelsymmetrische Stellung gebracht und der Vorgang wird wiederholt (vgl. Abb. \ref{picbrechwinkel}).