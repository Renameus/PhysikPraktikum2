%
%
\subsection{Differentielle Energieverteilung}
Die Abbildung \ref{fig1} zeigt trotz der verhältnismäßig wenigen Messwertpaaren
um das Maximum herum gut erkennbar die Energieverteilung der Elektronen und auch
$K=1$ V lässt mit einer gewissen Toleranz bestimmen. Das diese Kurve sich im Gegensatz 
zu Abbildung \ref{fig2} dafür eignet, liegt an dem Verhältnis a / $\overline w$, 
welches bei Raumtemperatur weit von dem in diesem Teil des Versuches störenden 
Bereiches der Stoßwahrscheinlichkeit entfernt ist, während es mit 1481 bei 
$T=147^{\circ}$ C im diesem Bereich von 1000 bis 4000 liegt. 
Der grundsätzliche Kurvenverlauf beschreibt die Änderung der Elektronenmenge, welche
die Auffängerelektrode erreicht. Bei der relativ niedrigen Temperatur kommt es aufgrund 
des erwähnten Weglängenverhältnisses kaum zu Stößen, sodass die Elektronen bei niedriger 
Gegenspannung fast alle die Auffängerelektrode erreichen. Steigt die Gegenspannung nun
weiter an, so wird bei betragsmäßig gleicher Beschleunigungspannung das Maximum der 
differentiellen Energieverteilung erwartet. Bezieht man das Kontaktpotential mit ein,
so muss das Maximum schon früher erreicht werden. Auch dannach gibt es weiter weniger Elektronen,
die die Auffängerelektrode erreichen, aber die Menge ändert sich nicht mehr so stark, die
Kurve flacht ab. Die Abbildung \ref{fig2} lässt keinen solchen Verlauf mehr erkennen, es
kommt zu inelastischen Stößen, die Elektronenernergieverteilung verbreitert sich, es  
lässt sich kein Kontaktpotential mehr ablesen. 
% ab einer Abbremsspannung von ungefähr 5 V
Außerdem fällt auf, dass  kaum noch ein Auffängerstrom 
registriert wird, was an der geringen Menge der Elektronen liegt, welche die Auffängerelektrode 
erreichen. Dem zugrunde liegt, dass zum einen die Temperatur groß genug ist, sodass es genug 
Quecksilberatome zum Anregen gibt, und zum anderen, dass die Elektronen genügend Energie
besitzen um diese Atome anzuregen.
\\
Insgesamt ließe sich die Genauigkeit noch steigern, wenn man $\Delta U$ kleiner wählte und
so einen noch genaueren Kurvenverlauf erhielte. Da aber viele weitere Ungenauigkeiten, wie
die ständige Änderung der Temperatur in einem sich aufheizenden Raum oder die Ungenauigkeit
des X-Y-Schreibers durch Liniendicke, Stiftrucklern oder ein bewegungssensitives 
Kabel, die Experimentellen Ergebnisse deutlich beeinflussten, wären kaum genauere 
Werte zu erwarten gewesen.

\subsection{Franck-Hertz-Kurve}
Die Maxima der Kurve ließen sich bei der gewählten Temperatur gut ablesen, auch wenn 
die Ausprägung mit größerem $n$ deutlich abnahm. Die erste Anregungsenergie mit (4,89$\pm$0,08) eV
passt gut zu dem Literaturwert von 4,9 eV für das Hg-Atom \cite{tafel}. Die bei diesem Vorgang
emittierte Wellenlänge beträgt (254$\pm$4) nm und befindet sich damit im Bereich der ultravioletten
Strahlung. \\
Die elastischen Stöße spielen hier nur eine untergeordnete Rolle, da der Energieverlust verhältnismäßig
gering ist. Bei den Fehlergrößenordnungen der darauf aufbauenden Größen kann dieser Energieverlust 
vernachlässigt werden. 

\subsection{Ionisation}
 Der Wert $U_i=11$ V, was einer Ionisationsenergie von 11 eV
 entspricht, stellt gegenüber dem Literaturwert von 10,43750 eV \cite{codatahgi} nur eine grobe Näherung dar, er beruht auf dem nur grob bestimmten $K$
und einem extrapolierten Wert aus einer ungenau aufgenommenen Kurve. Auch wenn noch keine
Asymptotik gegen eine Senkrechte zu erkennen ist, lässt dennoch 
der Verlauf gut erkennen, wann die Ionisationsprozesse einsetzen.  


