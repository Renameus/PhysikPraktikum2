%
%

Zunächst sind die mittleren freien Weglängen $\overline w$ für die benutzten Temperaturen
anzugeben. Gemäß der Formeln (7) und (8) lässt sich diese Größe leicht angeben:

\begin{table}[h]
\begin{center}
\begin{tabular}[c]{|c|c|c|} \hline
$\overline w$ in cm & $p_{s"at}$ in mBar & T in $^{\circ} K$\\ \hline
$5,8 \cdot 10^{-1}$ & 0,005 & 297,4\\ 
$2,73 \cdot 10^{-3}$ & 1,064 & 387,15 \\
$6,66 \cdot 10^{-4}$ & 4,353 & 420,5 \\ 
$1,64 \cdot 10^{-4}$ & 17,726 & 460 \\ \hline
\end{tabular}
\caption{Mittlere Weglängen}
\end{center}
\end{table}

Gemäß den Ausführungen in der Versuchsanleitung sollte zur Beobachtung des
Franck-Hertz-Effekts die mittlere freie Weglänge $\overline w$ mindestens um den Faktor
1000 kleiner als der Abstand a = 1 cm zwischen der Kathode und der Beschleunigungselektrode
sein. Somit sind die Messtemperaturen zwischen 420,5 $^{\circ} K$ und
460 $^{\circ} K$ prinzipiell für die Messung der Franck-Hertz-Kurve geeignet.

\subsection{Bestimmung der Energieverteilung der Elektronen}

Mit dem oben beschriebenen Aufbau sollte nun die integrale Energieverteilung
der beschleunigten Elektronen bestimmt werden. Hier ist zu beachten, dass die
Apparatur nur die Elektronen registriert, deren Geschwindigkeitskomponente $v_z$
in Feldrichtung die Ungleichung $E_z \geq e_0 U_A $ erfüllt.
Mit dem XY-Schreiber wurde der Auffängerstrom $I_A$ in Abhängigkeit von
der Bremsspannung $U_A$ bei zwei verschiedenen Temperaturen ($T_1$ = 23,9 $^{\circ} C$, $T_2$ = 147 $^{\circ} C$) aufgenommen.
Die Beschleunigerspannung $ U_B $ wurde auf 10 V eingestellt, der maximale so erreichbare Auffängerstrom betrug bei $T_1$
14,5 nA und bei $T_2$ waren es 1,15 nA.
Aus den Eichungen der Achsen ergeben sich folgende Faktoren:
\begin{table}[h]
\begin{center}
\begin{tabular}[c]{|c|c|c|} \hline
	 & $T_1$ & $T_2$\\ \hline
x-Achse & $0,045 \frac{V}{mm}$ & $0,034 \frac{V}{mm}$ \\ 
y-Achse & $0,093 \frac{nA}{mm}$ & $0,0077 \frac{nA}{mm}$ \\ \hline
\end{tabular}
\caption{Skalierung der Messwerte für die Energieverteilung}
\end{center}
\end{table}

Um aus den Kurven die gesuchte Anzahl derjenigen Elektronen zu erhalten,
deren Energie im Intervall [$E_z$,$E_z + \Delta E_z$] liegt, werden einige Wertepaare
$I_A(U_A)$,$U_A$ und $I_A(U_A+ \Delta U_A)$,$U_A$ abgelesen (siehe Tabelle im Anhang). Als $\Delta U_A$
wurde immer1 cm (auf der x-Achse) gewählt.(Graph im Anhang)

\begin{table}[h]
\begin{center}
\begin{tabular}[c]{|c|c||c|c|} \hline
$T_1$ x-Achse (U[V]) & $T_1$ y-Achse (I[nA]) & $T_2$ x-Achse (U[V]) & $T_2$ y-Achse (I[nA]) \\ \hline
0,00  & 14,5	 & 0,00 & 1,15\\
0,45  & 14,5	 & 0,34 & \\
0,90  & 14,4	 & 0,68 & \\
1,35  & 14,4	 & 1,02 & \\
1,80  & 14,3	 & 1,36 & \\
2,25  & 14,3	 & 1,70 & \\
2,70  & 14,2	 & 2,04 & \\
3,15  & 14,2	 & 2,38 & \\
3,60  & 14,1	 & 2,72 & \\
4,05  & 14,0	 & 3,06 & \\
4,50  & 14,0	 & 3,40 & \\
4,95  & 13,9	 & 3,74 & \\ 
5,40  & 13,8	 & 4,08 & \\
5,85  & 13,7	 & 4,42 & \\
6,30  & 13,6	 & 4,76 & \\
6,75  & 13,5	 & 5,10 & \\
7,20  & 13,1	 & 5,44 & \\
7,65  & 12,7	 & 5,78 & \\
8,10  & 12,1	 & 6,12 & \\
8,55  & 11,3	 & 6,46 & \\
9,00  & 10,1	 & 6,80 & \\
9,45  & 3,7	 & 7,14 & \\
9,90  & 0,2	 & 7,48 & \\ 
10,35& 0,1	 & 7,82 & \\ \hline 
\end{tabular}
\caption{Skalierung der Messwerte für die Energieverteilung}
\end{center}
\end{table}


Die Anzahl der Elektronen ist per Definition direkt aus der Stromstärke abzulesen,
da 1 A gleich 1 $\frac{C}{s}$ ist.
Da hier statt der tatsächlichen Ströme nur Differenzen aufgetragen wurden,
spiegelt das entstandene Diagramm die Änderung (Steigung) der mit dem xy-
Schreiber aufgenommenen Kurve wieder.
Als erstes betrachten wir die Kurve bei $23,9^\circ C$. In dieser ist der Betrag der (negativen)
Steigung von Beginn an sehr klein und nimmt dann im letzten Drittel
deutlich zu, bis die Kurve recht abrupt fast senkrecht fällt. Das allerletzte Stück
ist dann wieder gerade.
Entsprechend verläuft die Kurve des daraus erstellten Diagramms zunächst
recht flach um dann stark anzusteigen und am Ende wieder auf null zu fallen.
Physikalisch stellt sich eine Erklärung für den Kurvenverlauf (der vom xy-
Schreiber aufgenommenen Kurve) wie folgt dar:
Der Auffängerstrom ist zu Beginn maximal,
da die Elektronen kein Hindernis zu überwinden haben. Bei steigender Bremsspannung
können immer weniger Elektronen diese überwinden (aufgrund ihrer
zu geringen Geschwindigkeit in Feldrichtung). Ab einem bestimmten Punkt ist das hindernde
Potential groß genug um sämtliche Elektronen aufzuhalten.
Der besonders starke Abfall gegen Ende der
Kurve ist damit zu erklären, dass die Wahrscheinlichkeit für hohe Geschwindigkeiten
schnell stark abnimmt (siehe Fermi-Dirac-Verteilung).
Der Verlauf der Kurve für $147^\circ C$ sieht etwas anders aus. Hier ist ein kurzes,
steil fallendes Stück am Anfang zu sehen, während der Großteil fast gerade verläuft. Am
Ende geht auch hier der Strom auf null, jedoch sind alle Änderungen weniger
abrupt.
