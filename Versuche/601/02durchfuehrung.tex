%
%picaufbau
\subsection{Versuchsaufbau}
	\begin{figure}[h]
		\begin{center}
		\includegraphics[scale=0.3]{picaufbau.jpg}
		\caption{Versuchsschaltung [1]}
		\label{picaufbau}
		\end{center}	
	\end{figure}
Der Versuch wird wie in Abbildung \ref{picaufbau} aufgebaut, sodass
die Aufnahmen für die folgenden Versuchsteile aufgezeichnet werden können.
\subsection{Betimmung der Energieverteilung der beschleunigten Elektronen}
Nachdem der X-Y-Schreiber auf das aufzuzeichnende Intervall kalibriert ist, wird
bei konstantem Beschleunigungsstrom und
bei Zimmertemperatur, beziehungsweise bei erhöhter Temperatur
jeweils der Auffängerstrom in Abhängigkeit der Abbremsspannung aufgezeichnet.
\subsection{Bestimmung der Franck-Hertz-Kurve}
Nachdem der X-Y-Schreiber auf das aufzuzeichnende Intervall kalibriert ist, wird
bei passender Temperatur und konstanter Abbremsspannung der Auffängerstrom in 
Abhängigkeit der Beschleunigungsspannung 
aufgezeichnet, sodass deutlich genug Maxima zu erkennen sind.
\subsection{Bestimmung der Ionisierungsspannung von Hg}
Nachdem der X-Y-Schreiber auf das aufzuzeichnende Intervall kalibriert ist, wird
bei hoher Abbremsspannung und erhöhter Temperatur der Auffängerstrom in Abhängigkeit 
der Beschleunigungsspannung aufgezeichnet.